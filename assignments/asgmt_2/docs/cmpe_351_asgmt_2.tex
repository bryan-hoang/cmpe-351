\documentclass[
  coursecode={CMPE 351},
  assignmentname={Assignment 2},
  studentnumber=20053722,
  name={Bryan Hoang (16bch1)},
  draft,
  % final,
  date=2020-03-07,
]{
  ltxanswer%
}

\usepackage{bch-style}
\usepackage[final]{minted}
\usepackage{lstfiracode}

\setmonofont{FiraCode}[
  Contextuals=Alternate,
]

\ActivateVerbatimLigatures{}

\begin{document}
  \section{Part 1: Image Classification using CNN (50 points)}

  The performance of the initial model on the training and testing datasets are \qty{75.0}{\percent} and \qty{75.1}{\percent}, respectively.

  The following table summarizes the parameters and structure of the initial model below:
  \begin{minted}{text}
===============================================================================
Layer (type:depth-idx)                   Output Shape              Param #
===============================================================================
ConvolutionalNeuralNetwork               --                        --
├─Conv2d: 1-1                            [32, 6, 28, 28]           456
├─MaxPool2d: 1-2                         [32, 6, 14, 14]           --
├─Conv2d: 1-3                            [32, 16, 10, 10]          2,416
├─MaxPool2d: 1-4                         [32, 16, 5, 5]            --
├─Linear: 1-5                            [32, 128]                 51,328
├─Linear: 1-6                            [32, 13]                  1,677
===============================================================================
Total params: 55,877
Trainable params: 55,877
Non-trainable params: 0
Total mult-adds (M): 20.87
===============================================================================
Input size (MB): 0.39
Forward/backward pass size (MB): 1.65
Params size (MB): 0.22
Estimated Total Size (MB): 2.27
===============================================================================
  \end{minted}

  Steps to replicate the experiment:
  \begin{enumerate}
    \item Setting up the notebook by importing necessary models and packages. Notably, PyTorch will be used instead of Keras.
    \item Define the hyperparameters for the initial model.
    \item Define a custom \texttt{FashionProductImageDataset} class to load the data.
    \item Define a transformer for the images to all be the same size and normalize the values into the range of \([-1, 1]\).
    \item Load the dataset using the custom class created into step 3. Note that this is the directory structure of the files:
          \begin{minted}{text}
.
|-- data
|   |-- img/
|   |-- test.csv
|   `-- train.csv
`-- cmpe_351_asgmt_2.ipynb
    \end{minted}
    \item Test the flow of data through layers mimicing the initial model to validate dimensions with a trial.
    \item Initialize the model and summarize it.
    \item Train the model with 2 epochs.
    \item Evaluate the model.
  \end{enumerate}

  \section{Part 2: Improved Image Classification (50 points)}

  A 3rd convolutional layer was added to the base CNN model in addition to increasing the number of epochs from 2 to 4.

  The following table summarizes the parameters and structure of the improved model below:
  \begin{minted}{text}
===============================================================================
Layer (type:depth-idx)                   Output Shape              Param #
===============================================================================
ConvolutionalNeuralNetworkV2             --                        --
├─Conv2d: 1-1                            [32, 6, 28, 28]           456
├─MaxPool2d: 1-2                         [32, 6, 15, 15]           --
├─Conv2d: 1-3                            [32, 16, 11, 11]          2,416
├─MaxPool2d: 1-4                         [32, 16, 6, 6]            --
├─Conv2d: 1-5                            [32, 32, 2, 2]            12,832
├─MaxPool2d: 1-6                         [32, 32, 2, 2]            --
├─Linear: 1-7                            [32, 32]                  4,128
├─Linear: 1-8                            [32, 13]                  429
===============================================================================
Total params: 20,261
Trainable params: 20,261
Non-trainable params: 0
Total mult-adds (M): 22.58
===============================================================================
Input size (MB): 0.39
Forward/backward pass size (MB): 1.74
Params size (MB): 0.08
Estimated Total Size (MB): 2.22
===============================================================================
  \end{minted}

  The performance of the improved model on the testing datasets is \qty{81.6}{\percent} accuracy, respectively. This is approximately an \textbf{\qty{6.5}{\percent}} improvement in performance over the initial model on the testing dataset.

  Steps to replicate the experiment:
  \begin{enumerate}
    \item Setting up the notebook by importing necessary models and packages. Notably, PyTorch will be used instead of Keras.
    \item Define the hyperparameters for the initial model.
    \item Define a custom \texttt{FashionProductImageDataset} class to load the data.
    \item Define a transformer for the images to all be the same size and normalize the values into the range of \([-1, 1]\).
    \item Load the dataset using the custom class created into step 3.
          \begin{minted}{text}
.
|-- data
|   |-- img/
|   |-- test.csv
|   `-- train.csv
`-- cmpe_351_asgmt_2.ipynb
    \end{minted}
    \item Test the flow of data through layers mimicing the initial model to validate dimensions with a trial, \textbf{now with a 3rd convolutional layer}.
    \item Initialize the model and summarize it, \textbf{now with a 3rd convolutional layer}.
    \item Train the model, \textbf{now with 4 epochs}.
    \item Evaluate the model.
  \end{enumerate}
\end{document}
